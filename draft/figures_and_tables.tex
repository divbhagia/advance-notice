\documentclass{div}
\usepackage{pdflscape} % Package for landscape pages

\begin{document}
% \begin{table}[p]
%     \setstretch{1.15}
%     \caption{Descriptives by Notice Length}\label{tab_sum_stats}
%     \begin{threeparttable}
%     \begin{tabularx}{\textwidth}{p{0.245\textwidth}YYYp{0.15cm}YYY}
%     \toprule
%     & \multicolumn{3}{c}{Unbalanced} & & \multicolumn{3}{c}{Balanced} \\
%     & Short &  Long & Diff. & & Short &  Long & Diff. \\
%     & (1) & (2) & (2)-(1) & & (3) & (4) & (4)-(3) \\
%     \midrule 
%      Age              & 42.44  & 43.57  & 1.13*** &  & 43.04  & 43.01  & -0.03  \\
                  & (0.24) & (0.22) & (0.33)  &  & (0.24) & (0.22) & (0.33) \\
 Female           & 0.45   & 0.46   & 0.02    &  & 0.46   & 0.46   & -0.00  \\
                  & (0.01) & (0.01) & (0.02)  &  & (0.01) & (0.01) & (0.02) \\
 Married          & 0.59   & 0.63   & 0.04**  &  & 0.61   & 0.61   & -0.00  \\
                  & (0.01) & (0.01) & (0.02)  &  & (0.01) & (0.01) & (0.02) \\
 Black            & 0.10   & 0.09   & -0.01   &  & 0.10   & 0.09   & -0.00  \\
                  & (0.01) & (0.01) & (0.01)  &  & (0.01) & (0.01) & (0.01) \\
 College Degree   & 0.41   & 0.39   & -0.03*  &  & 0.40   & 0.40   & -0.00  \\
                  & (0.01) & (0.01) & (0.02)  &  & (0.01) & (0.01) & (0.02) \\
 Plant Closure    & 0.46   & 0.62   & 0.16*** &  & 0.54   & 0.54   & -0.00  \\
                  & (0.01) & (0.01) & (0.02)  &  & (0.01) & (0.01) & (0.02) \\
 Union Membership & 0.15   & 0.16   & 0.01    &  & 0.15   & 0.15   & 0.00   \\
                  & (0.01) & (0.01) & (0.01)  &  & (0.01) & (0.01) & (0.01) \\
 In Metro Area    & 0.84   & 0.82   & -0.01   &  & 0.83   & 0.83   & 0.00   \\
                  & (0.01) & (0.01) & (0.01)  &  & (0.01) & (0.01) & (0.01) \\
 Years of Tenure  & 7.12   & 9.18   & 2.06*** &  & 8.25   & 8.21   & -0.04  \\
                  & (0.15) & (0.16) & (0.22)  &  & (0.16) & (0.15) & (0.22) \\
 Log Earnings     & 6.54   & 6.56   & 0.03    &  & 6.54   & 6.55   & 0.00   \\
                  & (0.01) & (0.01) & (0.02)  &  & (0.01) & (0.01) & (0.02) \\
 Observations     & 1959   & 2216   &         &  & 1959   & 2216   &        \\
 \addlinespace[1ex]
%     \bottomrule
%     \end{tabularx}
%     %\begin{tablenotes}
%     %\item \textit{Note:} The sample consists of respondents from the Displaced Worker Supplement (DWS) for the years 1996-2020, who were between ages \agecutoff, had worked full-time for at least six months at their previous job, received health insurance from their former employer, and did not expect to be recalled. The sample excludes workers who were laid off in the year immediately preceding the survey. Short notice refers to a notice period of less than a month or between one and two months, while long notice refers to a notice period exceeding two months. Columns (1) and (2) present raw averages for the sample, while columns (3) and (4) show weighted averages, where the weights correspond to the inverse of the estimated probabilities of receiving short or long notice.      
%     %\end{tablenotes}
%     \end{threeparttable}
% \end{table}


\begin{landscape}
\begin{table}
     Age              & 42.44  & 43.57  & 1.13*** &  & 43.04  & 43.01  & -0.03  \\
                  & (0.24) & (0.22) & (0.33)  &  & (0.24) & (0.22) & (0.33) \\
 Female           & 0.45   & 0.46   & 0.02    &  & 0.46   & 0.46   & -0.00  \\
                  & (0.01) & (0.01) & (0.02)  &  & (0.01) & (0.01) & (0.02) \\
 Married          & 0.59   & 0.63   & 0.04**  &  & 0.61   & 0.61   & -0.00  \\
                  & (0.01) & (0.01) & (0.02)  &  & (0.01) & (0.01) & (0.02) \\
 Black            & 0.10   & 0.09   & -0.01   &  & 0.10   & 0.09   & -0.00  \\
                  & (0.01) & (0.01) & (0.01)  &  & (0.01) & (0.01) & (0.01) \\
 College Degree   & 0.41   & 0.39   & -0.03*  &  & 0.40   & 0.40   & -0.00  \\
                  & (0.01) & (0.01) & (0.02)  &  & (0.01) & (0.01) & (0.02) \\
 Plant Closure    & 0.46   & 0.62   & 0.16*** &  & 0.54   & 0.54   & -0.00  \\
                  & (0.01) & (0.01) & (0.02)  &  & (0.01) & (0.01) & (0.02) \\
 Union Membership & 0.15   & 0.16   & 0.01    &  & 0.15   & 0.15   & 0.00   \\
                  & (0.01) & (0.01) & (0.01)  &  & (0.01) & (0.01) & (0.01) \\
 In Metro Area    & 0.84   & 0.82   & -0.01   &  & 0.83   & 0.83   & 0.00   \\
                  & (0.01) & (0.01) & (0.01)  &  & (0.01) & (0.01) & (0.01) \\
 Years of Tenure  & 7.12   & 9.18   & 2.06*** &  & 8.25   & 8.21   & -0.04  \\
                  & (0.15) & (0.16) & (0.22)  &  & (0.16) & (0.15) & (0.22) \\
 Log Earnings     & 6.54   & 6.56   & 0.03    &  & 6.54   & 6.55   & 0.00   \\
                  & (0.01) & (0.01) & (0.02)  &  & (0.01) & (0.01) & (0.02) \\
 Observations     & 1959   & 2216   &         &  & 1959   & 2216   &        \\

\end{table}
\end{landscape}

\begin{figure}[t]\caption{IPW}
\vspace{-0.5em}
\centering
\begin{subfigure}{.525\textwidth}
\centering
\includegraphics{./../output/hazard_ipw.pdf}
\subcaption{Exit Rate}
\end{subfigure}
\begin{subfigure}{.45\textwidth}
\centering
\includegraphics{./../output/est_ipw.pdf}
\subcaption{Estimates}
\end{subfigure}
\vspace{-0.75em}
%\floatfoot{\textit{Note:} Short notice refers to a notice of less than 2 months, and long notice refers to a notice of more than 2 months. Panel A presents the weighted proportion of individuals exiting unemployment in each interval amongst those who were still unemployed at the beginning of the interval. Panel B presents the weighted proportion of individuals who are unemployed at the beginning of each interval. Error bars represent 90\% confidence intervals.}
\end{figure}

\begin{figure}[t]\caption{Raw}
\vspace{-0.5em}
\centering
\begin{subfigure}{.525\textwidth}
\centering
\includegraphics{./../output/hazard_raw.pdf}
\subcaption{Exit Rate}
\end{subfigure}
\begin{subfigure}{.45\textwidth}
\centering
\includegraphics{./../output/est_raw.pdf}
\subcaption{Survival Rate}
\end{subfigure}
\vspace{-0.75em}
%\floatfoot{\textit{Note:} Short notice refers to a notice of less than 2 months, and long notice refers to a notice of more than 2 months. Panel A presents the weighted proportion of individuals exiting unemployment in each interval amongst those who were still unemployed at the beginning of the interval. Panel B presents the weighted proportion of individuals who are unemployed at the beginning of each interval. Error bars represent 90\% confidence intervals.}
\end{figure}

\end{document}