\documentclass{div}


%%%%%%%%%%%%%%%%%%%% Locals 
\newcommand{\agecutoff}{21 to 64 } 
\newcommand{\outdir}{./../output}


\begin{document}

% Table 1: Descriptives by Notice Length
\begin{table}[p]
\setstretch{1.15}
\caption{Descriptives by Notice Length}\label{tab_sum_stats}
\begin{threeparttable}
\begin{tabularx}{\textwidth}{p{0.245\textwidth}YYYp{0.15cm}YYY}
\toprule
& \multicolumn{3}{c}{Unbalanced} & & \multicolumn{3}{c}{Balanced} \\
& Short &  Long & Diff. & & Short &  Long & Diff. \\
& (1) & (2) & (2)-(1) & & (3) & (4) & (4)-(3) \\
\midrule 
\input{\outdir/tab_sum_stats.tex} \addlinespace[1ex]
\bottomrule
\end{tabularx}
\begin{tablenotes}
\item \textit{Note:} The sample consists of respondents from the Displaced Worker Supplement (DWS) for the years 1996-2020, who were between ages of \agecutoff, had worked full-time for at least six months at their previous job, received health insurance from their former employer, and did not expect to be recalled. Short notice refers to a notice period of 1-2 months, while long notice refers to a notice period exceeding two months. Columns (1) and (2) present raw averages for the sample, while columns (3) and (4) show weighted averages, where the weights correspond to the inverse of the estimated probabilities of receiving short or long notice.      
\end{tablenotes}
\end{threeparttable}
\end{table}

% TABLE 2: Regression First 12 Weeks
\begin{table}[t]
\begin{threeparttable}
\caption{Observed Exit Rate -- Early in the Spell}\label{tab_init_hazard}
\begin{tabularx}{\textwidth}{p{0.225\textwidth}YYYY}
\toprule
& (1) & (2) & (3) & (4) \\
\midrule \addlinespace[1ex]
\multicolumn{5}{c}{\underline{\sc{Panel A. $\I\{\text{Unemployment duration $=0$ weeks}\}$}}} \\ \addlinespace[2ex]
\input{\outdir/tab_reg_init_hazardA.tex} \addlinespace[3ex]
\multicolumn{5}{c}{\underline{\sc{Panel B. $\I\{\text{Unemployment duration $\leq 12$ weeks}\}$}}} \\ \addlinespace[2ex]
\input{\outdir/tab_reg_init_hazardB.tex} \addlinespace[2ex]
Controls   &  No & Yes  & No & Yes \\
Weights   & No  & No   & Yes & Yes \\
\midrule
\input{\outdir/tab_reg_init_hazardC}
\bottomrule
\end{tabularx}
\begin{tablenotes}
\item \textit{Note:} The table presents estimates from linear regression models, where the main independent variable is an indicator variable that takes a value of 1 if the individual received a notice of more than 2 months, and 0 if they received a notice of 1-2 months. The dependent variable is an indicator for reporting an unemployment duration of 0 weeks (Panel A) or less than 12 weeks (Panel B). The weights are generated using inverse probability weighting (IPW). Robust standard errors are reported in the parenthesis. 
\end{tablenotes}
\end{threeparttable}
\end{table}

% FIGURE: Exit and Survival Rate --- Later in the Spell
\begin{figure}[t]\caption{Exit and Survival Rate --- Later in the Spell}
\vspace{-0.5em}
\centering
\begin{subfigure}{.525\textwidth}
\centering
\includegraphics{\outdir/fig_hazard_ipw.pdf}
\subcaption{Exit Rate}
\end{subfigure}
\begin{subfigure}{.45\textwidth}
\centering
\includegraphics{\outdir/fig_survival_ipw.pdf}
\subcaption{Survival Rate}
\end{subfigure}
\vspace{-0.75em}
\floatfoot{\textit{Note:} Short notice refers to a notice of less than 2 months, and long notice refers to a notice of more than 2 months. Panel A presents the weighted proportion of individuals exiting unemployment in each interval amongst those who were still unemployed at the beginning of the interval. Panel B presents the weighted proportion of individuals who are unemployed at the beginning of each interval. Error bars represent 90\% confidence intervals.}
\end{figure}

% TABLE: Baseline Estimates
\begin{table}[t]
\begin{threeparttable}
\caption{Estimation Results}\label{tab_baseline_estimates}
\begin{tabularx}{\linewidth}{Yp{0.5\textwidth}YY}
\toprule
Parameter & Explanation & Estimate & SE \\
\midrule  \addlinespace[1ex]
\multicolumn{4}{l}{\textit{Panel A: Estimated Parameters}}  \\ \addlinespace[1ex]
\input{\outdir/tab_baseline_estsA} \addlinespace[1ex]
\multicolumn{4}{l}{\textit{Panel B: Duration Dependence}}  \\ \addlinespace[1ex]
\input{\outdir/tab_baseline_estsB}   \\ 
\end{tabularx}
\begin{tabularx}{\linewidth}{p{1cm}XX}
\multicolumn{3}{l}{\textit{Hansen-Sargan Test}}  \\ \addlinespace[1ex]
\input{\outdir/tab_baseline_estsC}   
\bottomrule
\end{tabularx}
\begin{tablenotes}
\item \textit{Note:} The table presents estimates from the Mixed Hazard model. The first weighted moment is normalized to one, and structural duration dependence is specified by equation (\ref{eqn_functional_form}). Panel A shows the estimated parameters from the model, and panel B presents structural hazards implied by the estimated parameters. The standard errors for the structural hazards are calculated using the delta method.
\end{tablenotes}
\end{threeparttable}
\end{table}

% FIGURE: Baseline Estimates
\begin{figure}[t]\caption{Baseline Estimates}\label{fig_baseline_estimates}
\centering
%\vspace{-1.25em}
\begin{subfigure}{.49\linewidth}
\raggedleft
\includegraphics{\outdir/fig_baseline_estsA}
\subcaption{Structural Hazard}
\end{subfigure} \hfill
\begin{subfigure}{.49\linewidth}
\includegraphics{\outdir/fig_baseline_estsB}
\subcaption{Average Type}
\end{subfigure} 
\vspace{-0.75em}
\floatfoot{\textit{Note:} The solid blue line in panel A presents estimates for structural hazards as implied by the estimated parameters in panel A of Table \ref{tab_baseline_estimates}. The dotted black line in panel A presents the observed exit rate from the data, averaged across workers with short and long notice. Panel B presents the implied average type at each duration for those with short and long notice. Error bars represent 90\% confidence intervals.}
\end{figure}

% FIGURE: Calibration
\begin{figure}[t]\caption{Calibration of the Search Model}\label{fig_calibration}
\centering
\begin{subfigure}{.475\textwidth}
\includegraphics{\outdir/fig_calib_offer}
\subcaption{Offer Arrival Rate}
\end{subfigure} 
\begin{subfigure}{.475\textwidth}
\includegraphics{\outdir/fig_calib_search}
\subcaption{Search Effort}
\end{subfigure}
\floatfoot{\textit{Notes}: The figure presents the search effort and offer arrival rate from the calibration of the search model, assuming no heterogeneity (dashed black line) and assuming two types of workers (solid blue line). The search effort is averaged over two types of workers.}
\end{figure}


%%%%%%%%%%%%%%%%%%%%%%%%%%%%%%%%%%%%%%%%%%%%%%
%%%% Begin Appendix                             
%%%%%%%%%%%%%%%%%%%%%%%%%%%%%%%%%%%%%%%%%%%%%%

\clearpage
\setcounter{table}{0}
\setcounter{figure}{0}
\renewcommand{\thetable}{B\arabic{table}}
\renewcommand{\thefigure}{B\arabic{figure}}

% TABLE: Calibration Details
\begin{table}[h]
\begin{threeparttable}
\caption{Calibration Parameters for the Search Model}\label{tab_calib_details}
\begin{tabularx}{\textwidth}{p{0.65\textwidth}c}
\toprule
Parameter & Value \\
\midrule
Length of each period & 12 Weeks \\
Discount factor $\beta$ & 0.985 \\
Relative risk aversion $\sigma$ & 1.75 \\
Per period wages $w$ & 1 \\
Annuity Payments & 0.1 \\
Unemployment benefits & 0.5 \\
Benefit exhaustion $D_B$ & 3 \\
Search cost parameter $\rho$ & 1 \\
Search cost parameter $\theta$ & 50 \\
First period arrival rate $\delta(1)$ & 1 \\
\bottomrule
\end{tabularx}
\begin{tablenotes}
\item \textit{Note:} The table presents the parameters used for calibrating the search model in Section \ref{sec_search_model}.
\end{tablenotes}
\end{threeparttable}
\end{table}

% FIGURE: Calibration Fit
\begin{figure}[t]\caption{Search Model Calibration: Fit}\label{fig_calib_fit}
\centering
\begin{subfigure}{.49\linewidth}
\raggedleft
\includegraphics{\outdir/fig_search_model_fitA}
\subcaption{Observed Hazard}
\end{subfigure} \hfill
\begin{subfigure}{.49\linewidth}
\includegraphics{\outdir/fig_search_model_fitB}
\subcaption{Structural Hazard}
\end{subfigure} 
\vspace{-0.75em}
\floatfoot{\textit{Note:} The figure displays the fit of the search model for the two calibration exercises described in the text. Panel A shows the observed exit rate in the data (solid black line) alongside the corresponding fitted values obtained from calibrating the search model without heterogeneity (dotted red line) and with two types of workers (dashed blue line). Panel B displays the estimated structural hazard from the Mixed Hazard (MH) model (solid black line) and the fitted structural hazard from calibrating the search model with two types of workers (dashed blue line).}
\end{figure}

%%%%%%%%%%%%%%%%%%%% Appendix: Data
\setcounter{table}{0}
\setcounter{figure}{0}
\renewcommand{\thetable}{C\arabic{table}}
\renewcommand{\thefigure}{C\arabic{figure}}

% TABLE: Sample selection
\begin{table}[p]
\begin{threeparttable}
\caption{Sample Selection}\label{tab_sample}
\begin{tabularx}{\textwidth}{p{9cm}Y}
\toprule
Sample condition & Observations  \\ \midrule
\input{\outdir/tab_sample_selection.tex}
\bottomrule
\end{tabularx}
\begin{tablenotes}
\item  \textit{Note:} The table shows the number of observations remaining at each step of sample selection. 
\end{tablenotes}
\end{threeparttable}
\end{table}

% TABLE: Comparison of sample to DWS & CPS
\begin{table}[t]
\begin{threeparttable}
\centering
\caption{Comparison of the analytical sample to all individuals in the Displaced Worker Supplement (DWS) and the Current Population Survey (CPS)}\label{tab_cps_comparison}
\begin{tabularx}{\textwidth}{p{5cm}YYY}
\toprule
& Sample & DWS & CPS \\
& (1) & (2) & (3) \\
\midrule
\input{\outdir/tab_cps_comp.tex} 
\bottomrule
\end{tabularx}
\begin{tablenotes}
\item  \textit{Note:}  All samples are restricted to individuals between the ages of \agecutoff and pertain to years 1996-2020. Column (1) includes individuals from the DWS who worked full-time for at least six months and were provided health insurance at their lost job, did not expect to be recalled, and received a layoff notice of 1-2 months or greater than 2 months. Columns (2) and (3) include all individuals in the DWS and the monthly CPS, respectively.
\end{tablenotes}
\end{threeparttable}
\end{table}

% FIGURE: PS Balance
\begin{figure}[t]\caption{Assessing Overlap of Propensity Score Distributions}\label{fig_ps_bal}
\includegraphics{\outdir/fig_ps_balance}
\floatfoot{\textit{Note:} The figure presents the density of estimated propensity scores for individuals with short and long notice separately.}
\end{figure}

% FIGURE: Length of Notice over Time
\begin{figure}[t]\caption{Length of Notice over Time}\label{fig_dyear_bal}
\centering
\includegraphics{\outdir/fig_long_notice_ts.pdf}
\floatfoot{\textit{Note:} The figure plots a 3-year moving average of the proportion of individuals who received a notice of more than
2 months amongst all individuals in the sample who were displaced in a given year.}
\end{figure}

% FIGURE: Industry and Occupation
\begin{figure}[p]\caption{Industry and Occupation of the Lost Job}\label{fig_occ_ind_bal}
\begin{subfigure}{0.51\textwidth}
\includegraphics[trim = 0.5in 0in 0in 0in]{\outdir/fig_ind_ub.pdf}
\caption{Industry, Unbalanced}
\end{subfigure} 
\begin{subfigure}{0.475\textwidth}
\includegraphics[trim = 0.5in 0in 0in 0in]{\outdir/fig_ind_bal.pdf}
\caption{Industry, Balanced}
\end{subfigure}
\begin{subfigure}{0.51\textwidth}
\includegraphics[trim = 0.5in 0in 0in 0in]{\outdir/fig_occ_ub.pdf}
\caption{Occupation, Unbalanced}
\end{subfigure} 
\begin{subfigure}{0.475\textwidth}
\includegraphics[trim = 0.5in 0in 0in 0in]{\outdir/fig_occ_bal.pdf}
\caption{Occupation, Balanced}
\end{subfigure}
\floatfoot{\textit{Note:} The figure presents the proportions of individuals whose displaced jobs were in specific industries (panels A and B) and occupations (panels C and D) among long-notice and short-notice workers in both the unbalanced and balanced samples. The error bars represent the 90\% confidence intervals.}
\end{figure}

% TABLE: Regression of wages on notice length
\begin{table}[t]
\begin{threeparttable}
\caption{Earnings at the Subsequent Job}\label{tab_reg_wages}
\begin{tabularx}{\textwidth}{p{0.225\textwidth}YYYY}
\toprule
& \multicolumn{4}{c}{Weekly Log Earnings} \\
& (1) & (2) & (3) & (4) \\
\midrule
\input{\outdir/tab_reg_wagesA.tex} \addlinespace[2ex]
Controls   &  No & Yes  & No & Yes \\
Weights   & No  & No   & Yes & Yes \\
\midrule
\input{\outdir/tab_reg_wagesB.tex}
\bottomrule
\end{tabularx}
\begin{tablenotes}
\item \textit{Note:} The table shows results from linear regressions of log weekly wages at the subsequent job on an indicator for receiving a notice of more than 2 months. The sample used is similar to the main analytical sample, but it excludes individuals who had not yet found employment at the time of the survey, had multiple jobs between their previous and current job, or had incomplete earnings information for other reasons. Robust standard errors are reported in the parenthesis.
\end{tablenotes}
\end{threeparttable}
\end{table}

% TABLE: Unemployment Insurance Take-up
\begin{table}[h]\caption{Unemployment Insurance Take-up}\label{tab_uiben_recd}
\begin{threeparttable}
\begin{tabularx}{\textwidth}{XYY}
\toprule
Duration & Observations & Recieved UI Benefits \\
\midrule
\input{\outdir/tab_uiben_recd.tex} 
\bottomrule
\end{tabularx}
\begin{tablenotes}
\item \textit{Notes}: This table reports the percentage of individuals in the baseline sample who reported receiving UI benefits by the duration of unemployment.
\end{tablenotes}
\end{threeparttable}
\end{table}

% Figure: Unemployment Exhaustion break at 26
\begin{figure}[p]\caption{Timing of Benefit Exhaustion}\label{fig_uiex_break}
\includegraphics{\outdir/fig_uiex_break.pdf}
\floatfoot{\textit{Note:} The figure presents the proportion of individuals who report having exhausted their UI benefits by the duration of unemployment. The sample is restricted to individuals in the main analytical sample who reported receiving UI benefits, and duration is binned in 12-week intervals.} 
\end{figure}

% FIGURE: Alternative Bins 4 & 9 weeks: DATA
\begin{figure}[p]\caption{Survival and Exit Rates with Alternative Bins}\label{fig_altbins}
\begin{subfigure}{.475\textwidth}
\centering
\includegraphics{\outdir/fig_4week_survival.pdf}
\subcaption{Survival Rate}
\end{subfigure}
\begin{subfigure}{.475\textwidth}
\centering
\includegraphics{\outdir/fig_4week_hazard.pdf}
\subcaption{Exit Rate}
\end{subfigure}
\begin{subfigure}{.475\textwidth}
\includegraphics{\outdir/fig_9week_survival.pdf}
\subcaption{Survival Rate}
\end{subfigure}
\begin{subfigure}{.475\textwidth}
\includegraphics{\outdir/fig_9week_hazard.pdf}
\subcaption{Exit Rate}
\end{subfigure}
\floatfoot{\textit{Note:} Unemployment duration is binned in 4-week intervals for panels A and B, while it is binned in 9-week intervals for panels C and D. Panel A and C present the proportion of individuals who are unemployed at the beginning of each interval. Panel B and D present the proportion of individuals exiting unemployment in each interval amongst those who were still unemployed at the beginning of the interval. Error bars represent 90\% confidence intervals.}
\end{figure}

%%%%%%%%%%%%%%%%%%%% Appendix: Robustness
\setcounter{table}{0}
\setcounter{figure}{0}
\renewcommand{\thetable}{D\arabic{table}}
\renewcommand{\thefigure}{D\arabic{figure}}


% FIGURE: Data and Estimates using the Unweighted Sample
\begin{figure}[p]\caption{Data and Estimates using the Unweighted Sample}\label{fig_robust_unwtd}
\vspace{-0.25cm}
\begin{subfigure}{.475\textwidth}
\centering
\includegraphics{\outdir/fig_hazard_unwtd.pdf}
\subcaption{Exit Rate}
\end{subfigure}
\begin{subfigure}{.475\textwidth}
\includegraphics{\outdir/fig_robust_unwtd.pdf}
\subcaption{Estimated Structural Hazard}
\end{subfigure}
\floatfoot{\textit{Note:} The figure presents data and estimates for the unweighted analytical sample. Panel A displays the exit rate from the data separately for long and short-notice workers. In panel B, the solid blue line shows the estimated structural hazard from the Mixed Hazard model using unweighted moments, the dashed red line shows the estimate using weighted moments, and the dotted black line represents the average exit rate for short and long-notice workers in the data.}
\end{figure}

% FIGURE: Estimates with Alternative Notice Categories
\begin{figure}[p]\caption{Estimates with Alternative Notice Categories}\label{fig_robust_notice_cat}
\begin{subfigure}{.475\textwidth}
\centering
\includegraphics{\outdir/fig_notcatsA.pdf}
\subcaption{No notice, 1-2 months, >2 months}
\end{subfigure}
\begin{subfigure}{.475\textwidth}
\includegraphics{\outdir/fig_notcatsB.pdf}
\subcaption{<1 month, 1-2 months, >2 months}
\end{subfigure}
\floatfoot{\textit{Note:} Panel A presents the estimated hazard for the sample consisting of individuals who received no notice, along with those who received notice periods of 1-2 months and >2 months, as in the main analytical sample. Panel B estimates are for individuals who received notice of <1 month, in addition to the groups from the main analytical sample. J-Stat refers to the Sargan-Hansen statistic for overidentifying restrictions.}
\end{figure}

% FIGURE: Non-Parametric Estimates
\begin{figure}[t]\caption{Non-Parametric Estimates}\label{fig_robust_ff}
\centering 
\includegraphics{\outdir/fig_robust_np}
\floatfoot{\textit{Note:} The figure compares non-parametric estimates of the structural hazard from the Mixed Hazard model with baseline estimates that assume a log-logistic functional form for the structural hazard. Error bars represent 90\% confidence intervals.}
\end{figure}

% FIGURE: Alternative Bins
\begin{figure}[p]\caption{Estimates with Unemployment Duration Binned in 9-Week Intervals}\label{fig_robust_altbins}
\includegraphics{\outdir/fig_robust_9week}
\floatfoot{\textit{Note:} The figure presents estimates from the Mixed Hazard model using data with unemployment duration binned in 9-week intervals. The solid blue line represents estimates for the structural hazard assuming the Log-Logistic functional form. The dashed red line represents non-parametric estimates, while the dotted black line represents the observed exit rate from the data.}
\end{figure}

% FIGURE: Binning Robustness, Hazard
\begin{figure}[p]\caption{Estimation on Binned Simulated Data: Hazard}\label{fig_sim_binA}
	\includegraphics{\outdir/fig_sim_binningA.pdf}
\floatfoot{\textit{Note}: The figure presents results from a robustness exercise evaluating the impact of binning data into intervals. I calculate the moments used for estimation from the Mixed Hazard model with 12 durations. These moments are then binned into intervals of lengths ranging from 1 to 4, and the model is estimated using the binned moments. Along with the estimates of the structural hazard (dark grey solid line), I present two additional quantities binned similarly for comparison. The dotted red line represents the true structural cumulative hazard of exiting in the specific interval, while the black dashed line represents the average observed hazard.}
\end{figure}

% FIGURE: Binning Robustness, Average Type
\begin{figure}[p]\caption{Estimation on Binned Simulated Data: Average Type}\label{fig_sim_binB}
	\includegraphics{\outdir/fig_sim_binningB.pdf}
\floatfoot{\textit{Note}: The figure presents results from a robustness exercise evaluating the impact of binning data into intervals. I calculate the moments used for estimation from the Mixed Hazard model with 12 durations. These moments are then binned into intervals of lengths ranging from 1 to 4, and the model is estimated using the binned moments. The plots represent the average type at each duration as implied by the estimated structural hazard.}
\end{figure}

%%%%%%%%%%%%%%%%%%%% Appendix: Generalization
\setcounter{table}{0}
\setcounter{figure}{0}
\renewcommand{\thetable}{E\arabic{table}}
\renewcommand{\thefigure}{E\arabic{figure}}

% FIGURE: Alternative Assumptions on Heterogeneity Distribution
\begin{figure}[p]\caption{Allow average type to vary}\label{fig_genHet}
\begin{subfigure}{.475\textwidth}
\includegraphics{\outdir/fig_extgen_hetA}
\subcaption{Residuals}
\end{subfigure} 
\begin{subfigure}{.475\textwidth}
\includegraphics{\outdir/fig_extgen_hetB}
\subcaption{10 lowest-residual estimates}
\end{subfigure} 
\floatfoot{\textit{Note:} The figure presents results from the estimation of a more generalized Mixed Hazard model, where the mean of the heterogeneity distribution for individuals with different lengths of notice is allowed to vary according to the parameter $\kappa_1$. Panel A presents the residuals from GMM estimation for different values of $\kappa_1$. Panel B presents the structural hazard estimates from the 10 best models with the lowest-valued residuals (light grey lines), compared to the baseline estimate (solid line) and the observed hazard in the data (dashed line).}
\end{figure} 

% FIGURE: Alternative Assumptions on Structural Hazards
\begin{figure}[p]\caption{Allow structural hazards after the first period to vary}\label{fig_genStr}
\makebox[\textwidth]{
\begin{subfigure}{.475\textwidth}
\includegraphics{\outdir/fig_extgen_strhazA.pdf}
\subcaption{Residuals}
\end{subfigure} 
\begin{subfigure}{.475\textwidth}
\includegraphics{\outdir/fig_extgen_strhazB.pdf}
\subcaption{10 lowest-residual estimates}
\end{subfigure} 
\floatfoot{\textit{Note:}  The figure presents results from the estimation of a more generalized Mixed Hazard model, where the structural hazard after the initial period for individuals with different lengths of notice is allowed to vary according to the parameter $\gamma$. Panel A presents the residuals from GMM estimation for different values of $\gamma$. Panel B presents the structural hazard estimates from the 10 best models with the lowest-valued residuals (light grey lines), compared to the baseline estimate (solid line) and the observed hazard in the data (dashed line).}
}
\end{figure} 


% FIGURE: Alternative Assumptions on Structural Hazards and Distribution of Heterogeneity
\begin{figure}[p]\caption{Alternative Assumptions on Structural Hazards and Heterogeneity Distribution}\label{fig_genHetStr}
\makebox[\textwidth]{
\begin{subfigure}{.475\textwidth}
\includegraphics{\outdir/fig_extgen_bothA.pdf}
\subcaption{Residuals}
\end{subfigure} \hspace{1em}
\begin{subfigure}{.475\textwidth}
\includegraphics{\outdir/fig_extgen_bothB.pdf}
\subcaption{25 lowest-residual estimates}
\end{subfigure} 
\floatfoot{\textit{Note:}  The figure presents results from the estimation of a more generalized Mixed Hazard model. The mean of the heterogeneity distribution for individuals with different lengths of notice is allowed to vary according to the parameter $\kappa_1$. The structural hazard after the initial period for individuals with different lengths of notice is allowed to vary according to the parameter $\gamma$. Panel A presents the residuals from GMM estimation for different values of $\kappa_1$ and $\gamma$. Panel B presents the structural hazard estimates from the 25 best models with the lowest-valued residuals (light grey lines), compared to the baseline estimate (solid line) and the observed hazard in the data (dashed line).} 
}
\end{figure}

%%%%%%%%%%%%%%%%%%%% Appendix: Search Model
\setcounter{table}{0}
\setcounter{figure}{0}
\renewcommand{\thetable}{F\arabic{table}}
\renewcommand{\thefigure}{F\arabic{figure}}


% FIGURE: Simulation: Average Estimate
\begin{figure}[h]\caption{Simulation: Average Estimate}\label{fig_simulation_average}
\includegraphics{\outdir/fig_sm_sim_avgpred.pdf}
\floatfoot{\textit{Note:}  The solid blue line presents the average estimate from 1000 simulations of the search model. The dashed red line presents the structural duration dependence $\E [h(d|\nu)]$ implied by the model. While the dotted black line presents the observed structural duration dependence $\E [h(d|\nu)|D \geq d]$ implied by the model.}
\end{figure}

% FIGURE: Simulation: Distribution of Estimates
\begin{figure}[p]\caption{Estimates using Simulated Data from the Search Model}\label{fig_simulation_distribution} 
\begin{subfigure}{.4\textwidth}
\includegraphics{\outdir/fig_sm_sim_dist1.pdf}
\subcaption{$\psi(1)$}
\end{subfigure} 
\begin{subfigure}{.4\textwidth}
\includegraphics{\outdir/fig_sm_sim_dist2.pdf}
\subcaption{$\psi(2)$}
\end{subfigure} 
\begin{subfigure}{.4\textwidth} 
\includegraphics{\outdir/fig_sm_sim_dist3.pdf}
\subcaption{$\psi(3)$}
\end{subfigure} 
\begin{subfigure}{.4\textwidth}
\includegraphics{\outdir/fig_sm_sim_dist4.pdf}
\subcaption{$\psi(4)$}
\end{subfigure}
\floatfoot{\textit{Note:}  The figure presents the normalized distribution of structural duration dependence estimated on simulated data from the search model. The vertical lines represent the mean of the distribution for each structural hazard. Standard normal density is overlaid for reference.}
\end{figure}

\end{document}